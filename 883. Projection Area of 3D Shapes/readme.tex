\documentclass[12pt]{article}
\usepackage{amsmath}
\usepackage{listings}
\usepackage{xcolor}
\usepackage{graphicx}

\lstset{
  basicstyle=\ttfamily\small,  % Estilo básico
  breaklines=true,  % Permite la ruptura de líneas
  numbers=left,  % Números de línea a la izquierda
  numberstyle=\tiny,  % Estilo de los números de línea
  frame=tb,  % Marco en la parte superior e inferior
  columns=fullflexible,  % Columnas flexibles para manejar espacios
  showstringspaces=false,  % No mostrar espacios en cadenas
  commentstyle=\color{gray},  % Estilo de los comentarios
  keywordstyle=\color{blue},  % Estilo de las palabras clave
  identifierstyle=\color{violet},  % Estilo de los identificadores
  stringstyle=\color{red}  % Estilo de las cadenas de texto
}


\title{Reporte de Código: 238. Product of Array Except Self}
\author{Christian Eulogio Sanchez \\ 320196872}
\date{31-08-2023}

\begin{document}

\maketitle

\section{Reporte}

\subsection{¿Cómo logré resolver el problema?}
Entender el problema era la clave. Estás buscando el área proyectada de una cuadrícula tridimensional en los planos XY, YZ, y ZX. Por lo tanto, debes evaluar la cuadrícula desde tres ángulos diferentes.

\subsection{Estructuras de Datos Utilizadas}
Usé una matriz 2D (lista de listas en Python) para representar la cuadrícula. Las variables individuales \texttt{xy\_proj}, \texttt{yz\_proj}, y \texttt{zx\_proj} se usaron para almacenar las áreas de las proyecciones en los planos XY, YZ y ZX, respectivamente.

\subsection{Razón para la Solución}
Para \texttt{xy\_proj}, simplemente iteré a través de la cuadrícula y conté las celdas con un valor mayor que 0. \\
Para \texttt{yz\_proj} y \texttt{zx\_proj}, hallé la altura máxima en cada fila y columna respectivamente y las sumé.

\subsection{Dificultades}
La mayor dificultad fue entender bien el problema. Una vez que captas la esencia, el resto es simplemente aplicar lógica básica y usar bucles para iterar a través de la matriz.

\section{Código Fuente}

A continuación se presenta el código fuente en Python que resuelve el problema:

\begin{lstlisting}[language=Python]

class Solution:
    def projectionArea(self, grid: List[List[int]]) -> int: 
        xy_proj = 0  # Area of projection on xy-plane
        yz_proj = 0  # Area of projection on yz-plane
        zx_proj = 0  # Area of projection on zx-plane
        
        

        n = len(grid)  # Get the dimensions of the grid
        
        # Calculate xy_proj
        for i in range(n):
            for j in range(n):
                if grid[i][j] > 0:
                    xy_proj += 1
        
        # Calculate yz_proj
        for i in range(n):
            max_height_row = 0  # Max height in the current row
            for j in range(n):
                max_height_row = max(max_height_row, grid[i][j])
            yz_proj += max_height_row
        
        # Calculate zx_proj
        for j in range(n):
            max_height_col = 0  # Max height in the current column
            for i in range(n):
                max_height_col = max(max_height_col, grid[i][j])
            zx_proj += max_height_col
        
        # Total area of all three projections
        total_area = xy_proj + yz_proj + zx_proj
        
        return total_area


\end{lstlisting}

\section{Instrucciones para Ejecutar el Código en Python}

\begin{enumerate}
    \item \textbf{Guardar el Código:} Copia el código en un editor de texto y guárdalo con una extensión \texttt{.py}, por ejemplo, \texttt{solution.py}. (Agregue el archivo solution.py con el codigo para que no tengas que hacer este paso manualmente)
    
    \item \textbf{Abrir Terminal o Consola:} Abre una terminal o consola en tu sistema operativo.
    
    \item \textbf{Navegar al Directorio:} Usa el comando \texttt{cd} para navegar al directorio donde guardaste \texttt{solution.py}.
    \begin{lstlisting}[language=bash]
    cd path/to/directory
    \end{lstlisting}
    
    \item \textbf{Ejecutar el Código:} Una vez que estés en el directorio correcto, ejecuta el siguiente comando para iniciar el intérprete de Python:
    \begin{lstlisting}[language=bash]
    python solution.py
    \end{lstlisting}
    O si tienes Python 3.x instalado:
    \begin{lstlisting}[language=bash]
    python3 solution.py
    \end{lstlisting}
\end{enumerate}

\begin{figure}[h]
    \centering
    \includegraphics[width=1\linewidth]{Screensho.jpg}
    \caption{Codigo en LeetCode}
    \label{fig:enter-label}
\end{figure}

\end{document}
